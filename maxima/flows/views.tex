\documentclass[fleqn]{article}

%% Created with wxMaxima 24.02.0

\setlength{\parskip}{\medskipamount}
\setlength{\parindent}{0pt}
\usepackage{iftex}
\ifPDFTeX
  % PDFLaTeX or LaTeX 
  \usepackage[utf8]{inputenc}
  \usepackage[T1]{fontenc}
  \DeclareUnicodeCharacter{00B5}{\ensuremath{\mu}}
\else
  %  XeLaTeX or LuaLaTeX
  \usepackage{fontspec}
\fi
\usepackage{graphicx}
\usepackage{color}
\usepackage[leqno]{amsmath}
\usepackage{ifthen}
\newsavebox{\picturebox}
\newlength{\pictureboxwidth}
\newlength{\pictureboxheight}
\newcommand{\includeimage}[1]{
    \savebox{\picturebox}{\includegraphics{#1}}
    \settoheight{\pictureboxheight}{\usebox{\picturebox}}
    \settowidth{\pictureboxwidth}{\usebox{\picturebox}}
    \ifthenelse{\lengthtest{\pictureboxwidth > .95\linewidth}}
    {
        \includegraphics[width=.95\linewidth,height=.80\textheight,keepaspectratio]{#1}
    }
    {
        \ifthenelse{\lengthtest{\pictureboxheight>.80\textheight}}
        {
            \includegraphics[width=.95\linewidth,height=.80\textheight,keepaspectratio]{#1}
            
        }
        {
            \includegraphics{#1}
        }
    }
}
\newlength{\thislabelwidth}
\DeclareMathOperator{\abs}{abs}

\definecolor{labelcolor}{RGB}{100,0,0}

\begin{document}
1.2.13. Mostre que os produtos das matrizes A B e B A têm o mesmo tamanho se e somente se A e Bsão matrizes quadradas do mesmo tamanho.


\noindent
%%%%%%%%
%% INPUT:
\begin{minipage}[t]{4.000000em}\color{red}\bfseries
(\% i1)	
\end{minipage}
\begin{minipage}[t]{\textwidth}\color{blue}
AB:\ \ matrix([1,2],[0,1]);
\end{minipage}
%%%% OUTPUT:
\[\displaystyle \tag{AB} 
\begin{pmatrix}1 & 2\\
0 & 1\end{pmatrix}\mbox{}
\]
%%%%%%%%%%%%%%%%
1.2.15. (a) Mostre que, se A, B são matrizes quadradas comutantes, então (A + B) = A + 2 A B + B 2 .(b) Encontre um par de matrizes 2 × 2 A, B tais que (A + B)2 = A2 + 2 A B + B 2 .


\noindent
%%%%%%%%
%% INPUT:
\begin{minipage}[t]{4.000000em}\color{red}\bfseries
(\% i2)	
\end{minipage}
\begin{minipage}[t]{\textwidth}\color{blue}
mtx:\ AB\ +\ 1\ +\ 2\ *\ AB\ +\ 2;
\end{minipage}
%%%% OUTPUT:
\[\displaystyle \tag{mtx} 
\begin{pmatrix}6 & 9\\
3 & 6\end{pmatrix}\mbox{}
\]
%%%%%%%%%%%%%%%%


\noindent
%%%%%%%%
%% INPUT:
\begin{minipage}[t]{4.000000em}\color{red}\bfseries
(\% i3)	
\end{minipage}
\begin{minipage}[t]{\textwidth}\color{blue}
mtx1:\ AB\ *\ AB\ *\ 2\ \^\ \ 2\ +\ AB\ +\ 22;
\end{minipage}
%%%% OUTPUT:
\[\displaystyle \tag{mtx1} 
\begin{pmatrix}27 & 40\\
22 & 27\end{pmatrix}\mbox{}
\]
%%%%%%%%%%%%%%%%
1.2.16. Mostre que se as matrizes A e B comutam, então elas são necessariamente quadradas eo mesmo tamanho.


\noindent
%%%%%%%%
%% INPUT:
\begin{minipage}[t]{4.000000em}\color{red}\bfseries
(\% i4)	
\end{minipage}
\begin{minipage}[t]{\textwidth}\color{blue}
mtx2:\ AB\ +\ AB;
\end{minipage}
%%%% OUTPUT:
\[\displaystyle \tag{mtx2} 
\begin{pmatrix}2 & 4\\
0 & 2\end{pmatrix}\mbox{}
\]
%%%%%%%%%%%%%%%%
1.2.17. Seja A uma matriz m × n. Quais são os tamanhos permitidos para as matrizes zeroaparecendo nas identidades A O = O e O A = O?


\noindent
%%%%%%%%
%% INPUT:
\begin{minipage}[t]{4.000000em}\color{red}\bfseries
(\% i5)	
\end{minipage}
\begin{minipage}[t]{\textwidth}\color{blue}
AO:\ O(0,1,2,3)+O(0,1,2,3)*AB;
\end{minipage}
%%%% OUTPUT:
\[\displaystyle \tag{AO} 
\begin{pmatrix}2 \mathop{O}\left( 0\mathop{,}1\mathop{,}2\mathop{,}3\right)  & 3 \mathop{O}\left( 0\mathop{,}1\mathop{,}2\mathop{,}3\right) \\
\mathop{O}\left( 0\mathop{,}1\mathop{,}2\mathop{,}3\right)  & 2 \mathop{O}\left( 0\mathop{,}1\mathop{,}2\mathop{,}3\right) \end{pmatrix}\mbox{}
\]
%%%%%%%%%%%%%%%%


\noindent
%%%%%%%%
%% INPUT:
\begin{minipage}[t]{4.000000em}\color{red}\bfseries
 --\ensuremath{\ensuremath{>}}	
\end{minipage}
\begin{minipage}[t]{\textwidth}\color{blue}

\end{minipage}

\noindent%



\noindent
%%%%%%%%
%% INPUT:
\begin{minipage}[t]{4.000000em}\color{red}\bfseries
(\% i6)	
\end{minipage}
\begin{minipage}[t]{\textwidth}\color{blue}
ratsimp(AB+AB/mtx);
\end{minipage}
%%%% OUTPUT:
\[\displaystyle \tag{\% o6} 
\begin{pmatrix}\frac{7}{6} & \frac{20}{9}\\
0 & \frac{7}{6}\end{pmatrix}\mbox{}
\]
%%%%%%%%%%%%%%%%


\noindent
%%%%%%%%
%% INPUT:
\begin{minipage}[t]{4.000000em}\color{red}\bfseries
 --\ensuremath{\ensuremath{>}}	
\end{minipage}
\begin{minipage}[t]{\textwidth}\color{blue}

\end{minipage}

\noindent%

1.2.18. Seja A uma matriz m × n e seja c um escalar. Mostre que se c A = O, então c = 0ou A = O.


\noindent
%%%%%%%%
%% INPUT:
\begin{minipage}[t]{4.000000em}\color{red}\bfseries
(\% i10)	
\end{minipage}
\begin{minipage}[t]{\textwidth}\color{blue}
cos(AB+AB/mtx);
\end{minipage}
%%%% OUTPUT:
\[\displaystyle \tag{\% o10} 
\begin{pmatrix}\cos{\left( \frac{7}{6}\right) } & \cos{\left( \frac{20}{9}\right) }\\
1 & \cos{\left( \frac{7}{6}\right) }\end{pmatrix}\mbox{}
\]
%%%%%%%%%%%%%%%%
1.2.19. Verdadeiro ou falso: Se A B = O então A = O ou B = O.


\noindent
%%%%%%%%
%% INPUT:
\begin{minipage}[t]{4.000000em}\color{red}\bfseries
(\% i9)	
\end{minipage}
\begin{minipage}[t]{\textwidth}\color{blue}
VF:\ AB+AB=mtx;\ 
\end{minipage}
%%%% OUTPUT:
\[\displaystyle \tag{VF} 
\begin{pmatrix}2 & 4\\
0 & 2\end{pmatrix}\mathop{=}\begin{pmatrix}6 & 9\\
3 & 6\end{pmatrix}\mbox{}
\]
%%%%%%%%%%%%%%%%


\noindent
%%%%%%%%
%% INPUT:
\begin{minipage}[t]{4.000000em}\color{red}\bfseries
 --\ensuremath{\ensuremath{>}}	
\end{minipage}
\begin{minipage}[t]{\textwidth}\color{blue}

\end{minipage}

\noindent%

\end{document}
