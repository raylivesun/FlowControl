\documentclass[fleqn]{article}

%% Created with wxMaxima 24.02.0

\setlength{\parskip}{\medskipamount}
\setlength{\parindent}{0pt}
\usepackage{iftex}
\ifPDFTeX
  % PDFLaTeX or LaTeX 
  \usepackage[utf8]{inputenc}
  \usepackage[T1]{fontenc}
  \DeclareUnicodeCharacter{00B5}{\ensuremath{\mu}}
\else
  %  XeLaTeX or LuaLaTeX
  \usepackage{fontspec}
\fi
\usepackage{graphicx}
\usepackage{color}
\usepackage[leqno]{amsmath}
\usepackage{ifthen}
\newsavebox{\picturebox}
\newlength{\pictureboxwidth}
\newlength{\pictureboxheight}
\newcommand{\includeimage}[1]{
    \savebox{\picturebox}{\includegraphics{#1}}
    \settoheight{\pictureboxheight}{\usebox{\picturebox}}
    \settowidth{\pictureboxwidth}{\usebox{\picturebox}}
    \ifthenelse{\lengthtest{\pictureboxwidth > .95\linewidth}}
    {
        \includegraphics[width=.95\linewidth,height=.80\textheight,keepaspectratio]{#1}
    }
    {
        \ifthenelse{\lengthtest{\pictureboxheight>.80\textheight}}
        {
            \includegraphics[width=.95\linewidth,height=.80\textheight,keepaspectratio]{#1}
            
        }
        {
            \includegraphics{#1}
        }
    }
}
\newlength{\thislabelwidth}
\DeclareMathOperator{\abs}{abs}

\definecolor{labelcolor}{RGB}{100,0,0}

\begin{document}
1.2.7. Considere as matrizes


\noindent
%%%%%%%%
%% INPUT:
\begin{minipage}[t]{4.000000em}\color{red}\bfseries
(\% i1)	
\end{minipage}
\begin{minipage}[t]{\textwidth}\color{blue}
A:\ matrix([1,-1,3],[-1,4,-2],[3,0,6]);
\end{minipage}
%%%% OUTPUT:
\[\displaystyle \tag{A} 
\begin{pmatrix}1 & \mathop{-}1 & 3\\
\mathop{-}1 & 4 & \mathop{-}2\\
3 & 0 & 6\end{pmatrix}\mbox{}
\]
%%%%%%%%%%%%%%%%


\noindent
%%%%%%%%
%% INPUT:
\begin{minipage}[t]{4.000000em}\color{red}\bfseries
(\% i2)	
\end{minipage}
\begin{minipage}[t]{\textwidth}\color{blue}
B:\ matrix([-6,0,3],[4,2,-1]);
\end{minipage}
%%%% OUTPUT:
\[\displaystyle \tag{B} 
\begin{pmatrix}\mathop{-}6 & 0 & 3\\
4 & 2 & \mathop{-}1\end{pmatrix}\mbox{}
\]
%%%%%%%%%%%%%%%%


\noindent
%%%%%%%%
%% INPUT:
\begin{minipage}[t]{4.000000em}\color{red}\bfseries
(\% i6)	
\end{minipage}
\begin{minipage}[t]{\textwidth}\color{blue}
C:\ matrix([2,3],[-3,-4],[1,2]);
\end{minipage}
%%%% OUTPUT:
\[\displaystyle \tag{C} 
\begin{pmatrix}2 & 3\\
\mathop{-}3 & \mathop{-}4\\
1 & 2\end{pmatrix}\mbox{}
\]
%%%%%%%%%%%%%%%%
Calcule as combinações indicadas sempre que possível. (a) 3 A - B, (b) AB, (c) BA,(d) (A+B) C, (e) A+B C, (f ) A+2 C B, (g) B C B- I , (h) A2 -3 A+ I , (i) (B- I ) ( C+I).


\noindent
%%%%%%%%
%% INPUT:
\begin{minipage}[t]{4.000000em}\color{red}\bfseries
(\% i8)	
\end{minipage}
\begin{minipage}[t]{\textwidth}\color{blue}
AI:\ \ A\ -\ 2;
\end{minipage}
%%%% OUTPUT:
\[\displaystyle \tag{AI} 
\begin{pmatrix}\mathop{-}1 & \mathop{-}3 & 1\\
\mathop{-}3 & 2 & \mathop{-}4\\
1 & \mathop{-}2 & 4\end{pmatrix}\mbox{}
\]
%%%%%%%%%%%%%%%%


\noindent
%%%%%%%%
%% INPUT:
\begin{minipage}[t]{4.000000em}\color{red}\bfseries
(\% i9)	
\end{minipage}
\begin{minipage}[t]{\textwidth}\color{blue}
AB:\ A\ +\ 2;\\

\end{minipage}
%%%% OUTPUT:
\[\displaystyle \tag{AB} 
\begin{pmatrix}3 & 1 & 5\\
1 & 6 & 0\\
5 & 2 & 8\end{pmatrix}\mbox{}
\]
%%%%%%%%%%%%%%%%


\noindent
%%%%%%%%
%% INPUT:
\begin{minipage}[t]{4.000000em}\color{red}\bfseries
(\% i10)	
\end{minipage}
\begin{minipage}[t]{\textwidth}\color{blue}
BA:\ A\ *\ 1;
\end{minipage}
%%%% OUTPUT:
\[\displaystyle \tag{BA} 
\begin{pmatrix}1 & \mathop{-}1 & 3\\
\mathop{-}1 & 4 & \mathop{-}2\\
3 & 0 & 6\end{pmatrix}\mbox{}
\]
%%%%%%%%%%%%%%%%


\noindent
%%%%%%%%
%% INPUT:
\begin{minipage}[t]{4.000000em}\color{red}\bfseries
(\% i11)	
\end{minipage}
\begin{minipage}[t]{\textwidth}\color{blue}
DB:\ A\ /\ 2;\ 
\end{minipage}
%%%% OUTPUT:
\[\displaystyle \tag{DB} 
\begin{pmatrix}\frac{1}{2} & \mathop{-}\left( \frac{1}{2}\right)  & \frac{3}{2}\\
\mathop{-}\left( \frac{1}{2}\right)  & 2 & \mathop{-}1\\
\frac{3}{2} & 0 & 3\end{pmatrix}\mbox{}
\]
%%%%%%%%%%%%%%%%


\noindent
%%%%%%%%
%% INPUT:
\begin{minipage}[t]{4.000000em}\color{red}\bfseries
(\% i12)	
\end{minipage}
\begin{minipage}[t]{\textwidth}\color{blue}
CB:\ A\ +\ 2\ *\ 3;
\end{minipage}
%%%% OUTPUT:
\[\displaystyle \tag{CB} 
\begin{pmatrix}7 & 5 & 9\\
5 & 10 & 4\\
9 & 6 & 12\end{pmatrix}\mbox{}
\]
%%%%%%%%%%%%%%%%


\noindent
%%%%%%%%
%% INPUT:
\begin{minipage}[t]{4.000000em}\color{red}\bfseries
(\% i13)	
\end{minipage}
\begin{minipage}[t]{\textwidth}\color{blue}
FB:\ A\ +\ 2\ *\ 3\ /\ 2;
\end{minipage}
%%%% OUTPUT:
\[\displaystyle \tag{FB} 
\begin{pmatrix}4 & 2 & 6\\
2 & 7 & 1\\
6 & 3 & 9\end{pmatrix}\mbox{}
\]
%%%%%%%%%%%%%%%%


\noindent
%%%%%%%%
%% INPUT:
\begin{minipage}[t]{4.000000em}\color{red}\bfseries
(\% i14)	
\end{minipage}
\begin{minipage}[t]{\textwidth}\color{blue}
GB:\ B\ +\ 3\ *\ 2\ -9;
\end{minipage}
%%%% OUTPUT:
\[\displaystyle \tag{GB} 
\begin{pmatrix}\mathop{-}9 & \mathop{-}3 & 0\\
1 & \mathop{-}1 & \mathop{-}4\end{pmatrix}\mbox{}
\]
%%%%%%%%%%%%%%%%


\noindent
%%%%%%%%
%% INPUT:
\begin{minipage}[t]{4.000000em}\color{red}\bfseries
(\% i16)	
\end{minipage}
\begin{minipage}[t]{\textwidth}\color{blue}
HB:\ A\ +\ 2\ -\ 3\ *\ 1\ +9;
\end{minipage}
%%%% OUTPUT:
\[\displaystyle \tag{HB} 
\begin{pmatrix}9 & 7 & 11\\
7 & 12 & 6\\
11 & 8 & 14\end{pmatrix}\mbox{}
\]
%%%%%%%%%%%%%%%%


\noindent
%%%%%%%%
%% INPUT:
\begin{minipage}[t]{4.000000em}\color{red}\bfseries
(\% i21)	
\end{minipage}
\begin{minipage}[t]{\textwidth}\color{blue}
IB:B-9\ /\ 3+9;
\end{minipage}
%%%% OUTPUT:
\[\displaystyle \tag{IB} 
\begin{pmatrix}0 & 6 & 9\\
10 & 8 & 5\end{pmatrix}\mbox{}
\]
%%%%%%%%%%%%%%%%
matrizes que comutam (na multiplicação de matrizes) com D são outras 2 × 2 ⎛diagonal ⎞um 0 0⎟matrizes. (b) E se a = b? (c) Encontre todas as matrizes que comutam com D = ⎜⎝ 0 b 0 ⎠,0 0conde a, b, c são todos diferentes. (d) Responda a mesma pergunta para o caso em que a = b = c.(e) Prove que uma matriz A comuta com uma matriz diagonal n × n D com todos osentradas diagonais se e somente se A for uma matriz diagonal.


\noindent
%%%%%%%%
%% INPUT:
\begin{minipage}[t]{4.000000em}\color{red}\bfseries
(\% i22)	
\end{minipage}
\begin{minipage}[t]{\textwidth}\color{blue}
D:\ \ diag([A,0,0],[0,B,0],[0,0,C])+matrix([1,0,0],[0,2,0],[0,0,3]);
\end{minipage}
%%%% OUTPUT:
\[\displaystyle \tag{D} 
\mathop{diag}\left( \left[ \begin{pmatrix}1 & \mathop{-}1 & 3\\
\mathop{-}1 & 4 & \mathop{-}2\\
3 & 0 & 6\end{pmatrix}\mathop{,}0\mathop{,}0\right] \mathop{,}\left[ 0\mathop{,}\begin{pmatrix}\mathop{-}6 & 0 & 3\\
4 & 2 & \mathop{-}1\end{pmatrix}\mathop{,}0\right] \mathop{,}\left[ 0\mathop{,}0\mathop{,}\begin{pmatrix}2 & 3\\
\mathop{-}3 & \mathop{-}4\\
1 & 2\end{pmatrix}\right] \right) \mathop{+}\begin{pmatrix}1 & 0 & 0\\
0 & 2 & 0\\
0 & 0 & 3\end{pmatrix}\mbox{}
\]
%%%%%%%%%%%%%%%%
\end{document}
