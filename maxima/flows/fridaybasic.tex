\documentclass[fleqn]{article}

%% Created with wxMaxima 24.02.0

\setlength{\parskip}{\medskipamount}
\setlength{\parindent}{0pt}
\usepackage{iftex}
\ifPDFTeX
  % PDFLaTeX or LaTeX 
  \usepackage[utf8]{inputenc}
  \usepackage[T1]{fontenc}
  \DeclareUnicodeCharacter{00B5}{\ensuremath{\mu}}
\else
  %  XeLaTeX or LuaLaTeX
  \usepackage{fontspec}
\fi
\usepackage{graphicx}
\usepackage{color}
\usepackage[leqno]{amsmath}
\usepackage{ifthen}
\newsavebox{\picturebox}
\newlength{\pictureboxwidth}
\newlength{\pictureboxheight}
\newcommand{\includeimage}[1]{
    \savebox{\picturebox}{\includegraphics{#1}}
    \settoheight{\pictureboxheight}{\usebox{\picturebox}}
    \settowidth{\pictureboxwidth}{\usebox{\picturebox}}
    \ifthenelse{\lengthtest{\pictureboxwidth > .95\linewidth}}
    {
        \includegraphics[width=.95\linewidth,height=.80\textheight,keepaspectratio]{#1}
    }
    {
        \ifthenelse{\lengthtest{\pictureboxheight>.80\textheight}}
        {
            \includegraphics[width=.95\linewidth,height=.80\textheight,keepaspectratio]{#1}
            
        }
        {
            \includegraphics{#1}
        }
    }
}
\newlength{\thislabelwidth}
\DeclareMathOperator{\abs}{abs}

\definecolor{labelcolor}{RGB}{100,0,0}

\begin{document}
Por exemplo, se adicionarmos - 2 vezes a primeira linha da matriz aumentada (1.16) à segundalinha, o resultado é o vetor linha


\noindent
%%%%%%%%
%% INPUT:
\begin{minipage}[t]{4.000000em}\color{red}\bfseries
(\% i2)	
\end{minipage}
\begin{minipage}[t]{\textwidth}\color{blue}
A:\ -2\ +\ matrix([1,2,1,2])\ +\ matrix([2,6,1,7])=matrix([0,2-1,3]);
\end{minipage}
%%%% OUTPUT:
\[\displaystyle \tag{A} 
\begin{pmatrix}1 & 6 & 0 & 7\end{pmatrix}\mathop{=}\begin{pmatrix}0 & 1 & 3\end{pmatrix}\mbox{}
\]
%%%%%%%%%%%%%%%%
O resultado pode ser reconhecido como a segunda linha da matriz aumentada modificada


\noindent
%%%%%%%%
%% INPUT:
\begin{minipage}[t]{4.000000em}\color{red}\bfseries
(\% i3)	
\end{minipage}
\begin{minipage}[t]{\textwidth}\color{blue}
B:\ matrix([1,2,1+2],[0,2,-1+3],[1,1,4]);
\end{minipage}
%%%% OUTPUT:
\[\displaystyle \tag{B} 
\begin{pmatrix}1 & 2 & 3\\
0 & 2 & 2\\
1 & 1 & 4\end{pmatrix}\mbox{}
\]
%%%%%%%%%%%%%%%%
que corresponde ao primeiro sistema equivalente (1.2). Quando a operação de linha elementar nº 1é executado, é fundamental que o resultado substitua a linha que está sendo adicionada - e não a linhasendo multiplicado pelo escalar. Observe que a eliminação de uma variável em uma equação -neste caso, a primeira variável na segunda equação - equivale a fazer a sua entrada namatriz de coeficientes igual a zero.Chamaremos a entrada (1, 1) da matriz de coeficientes de primeiro pivô. O precisoa definição de pivô ficará clara à medida que prosseguirmos; o único requisito fundamental é que umo pivô deve ser sempre diferente de zero. Eliminando a primeira variável x da segunda e terceiraequações equivale a tornar todas as entradas da matriz na coluna abaixo do pivô iguais azero. Já fizemos isso com a entrada (2, 1) em (1.17). Para fazer a entrada (3, 1)igual a zero, subtraímos (ou seja, adicionamos -1 vezes) a primeira linha da última linha. O


\noindent
%%%%%%%%
%% INPUT:
\begin{minipage}[t]{4.000000em}\color{red}\bfseries
(\% i5)	
\end{minipage}
\begin{minipage}[t]{\textwidth}\color{blue}
C:\ matrix([1,2,1\^\ 2],[0,2,-1\^\ 3],[0,-1,3\^\ 1]);
\end{minipage}
%%%% OUTPUT:
\[\displaystyle \tag{C} 
\begin{pmatrix}1 & 2 & 1\\
0 & 2 & \mathop{-}1\\
0 & \mathop{-}1 & 3\end{pmatrix}\mbox{}
\]
%%%%%%%%%%%%%%%%
que corresponde ao sistema (1.3). O segundo pivô é a entrada (2, 2) desta matriz,que é 2 e é o coeficiente da segunda variável na segunda equação. Novamente, oo pivô deve ser diferente de zero. Usamos a operação elementar de adição de 21 do segundolinha para a terceira linha para tornar a entrada abaixo do segundo pivô igual a 0; o resultado é omatriz aumentada


\noindent
%%%%%%%%
%% INPUT:
\begin{minipage}[t]{4.000000em}\color{red}\bfseries
(\% i6)	
\end{minipage}
\begin{minipage}[t]{\textwidth}\color{blue}
N:\ matrix([1,2,1\^\ 2],[0,2,-1\^\ 3],[0,0,5/2\^\ 5/2]);
\end{minipage}
%%%% OUTPUT:
\[\displaystyle \tag{N} 
\begin{pmatrix}1 & 2 & 1\\
0 & 2 & \mathop{-}1\\
0 & 0 & \frac{5}{64}\end{pmatrix}\mbox{}
\]
%%%%%%%%%%%%%%%%
\end{document}
